\documentclass[letterpaper,11pt]{article}

% Encoding
\usepackage[utf8]{inputenc}

% Page structure
\usepackage{geometry}
\geometry{margin=2.54cm}
\usepackage[skins]{tcolorbox}
%\geometry{top=1.54cm}

% Titles
\usepackage{sectsty}
\allsectionsfont{\sffamily\mdseries\upshape}

% Headers and Footers
\usepackage{fancyhdr}
\pagestyle{fancy}

% Graphics, figures and plots
\usepackage{graphicx}
\usepackage{float}
\usepackage{pgfplots}
\usepackage{caption}
\usepackage{subcaption}
\usepackage{changepage}

% Etc
\usepackage{hyperref}

% Bibliography
\usepackage[numbers]{natbib}

%%%%%%%%%%
% Math
\usepackage{amsmath}
\usepackage{amsthm}
\usepackage{amsfonts}
\usepackage{amssymb}
\usepackage{amscd}
\usepackage{eurosym}
\usepackage{mathrsfs}
\usepackage{mathtools}
\usepackage{bm} % defines a com­mand \bm which makes its ar­gu­ment bold
\usepackage{bbm}

\usepackage{enumitem, hyperref}
\makeatletter
\def\namedlabel#1#2{\begingroup
    #2%
    \def\@currentlabel{#2}%
    \phantomsection\label{#1}\endgroup
}
\makeatother

%%%% Mathematical Environments
\theoremstyle{plain}
\newtheorem{theorem}{Theorem}[section]
\newtheorem{lemma}[theorem]{Lemma}
\newtheorem{corollary}[theorem]{Corollary}
\newtheorem{proposition}[theorem]{Proposition}
\newtheorem{conjecture}[theorem]{Conjecture}

\theoremstyle{definition}
\newtheorem{definition}[theorem]{Definition}
\newtheorem{example}[theorem]{Example}

\theoremstyle{remark}
\newtheorem{remark}[theorem]{Remark}
\newtheorem{note}[theorem]{Note}
\newtheorem{case}[theorem]{Case}

%%%%%%%%%%
% Pseudocode
\usepackage[ruled]{algorithm2e}
%\usepackage{algpseudocode}
%\usepackage{verbatim}

%%%%%%%%%%
% Operators
\DeclareMathOperator*{\argmax}{arg\,max}
\DeclareMathOperator*{\argmin}{arg\,min}
\DeclareMathOperator*{\Exp}{\mathbb{E}}
\DeclareMathOperator*{\Var}{\text{Var}}
\DeclareMathOperator*{\Cov}{\text{Cov}}

%%%%%%%%%%
% Paired -- \command* will automatically resize
\DeclarePairedDelimiter\paren{(}{)}           % (parentheses)
\DeclarePairedDelimiter\ang{\langle}{\rangle} % <angle brackets>
\DeclarePairedDelimiter\abs{\lvert}{\rvert}   % |absolute value|
\DeclarePairedDelimiter\norm{\lVert}{\rVert}  % ||norm||
\DeclarePairedDelimiter\bkt{[}{]}             % [brackets]
\DeclarePairedDelimiter\set{\{}{\}}           % {braces}

%%%%%%%%%%
% Macros
% Bidders
\newcommand{\bidders}[1][]{N}
\newcommand{\numbidders}[1][]{n}
% Goods
\newcommand{\goods}[1][]{G}
\newcommand{\numgoods}[1][]{g}
% Allocations
\newcommand{\alloc}[1][]{x_{#1}}
\newcommand{\allocvec}[1][]{\mathbf{\alloc}}
\newcommand{\intalloc}[1][]{{\hat{\alloc}}_{#1}}
\newcommand{\intallocvec}[1][]{\mathbf{\intalloc}}
% Valuations
\newcommand{\val}[1][]{v_{#1}}
\newcommand{\valvec}[1][]{\mathbf{\val}_{#1}}
\newcommand{\valz}[1][]{z_{#1}}
\newcommand{\valzvec}[1][]{\mathbf{\valz}_{#1}}
\newcommand{\valspace}[1][]{T_{#1}}
\newcommand{\valspacesize}[1][]{M_{#1}}
% Payment
\newcommand{\payment}[1][]{p_{#1}}
\newcommand{\paymentvec}[1][]{\mathbf{\payment}_{#1}}
% Virtual values (continuous and discrete)
\newcommand{\virval}[1][]{\varphi_{#1}}
\newcommand{\virvalvec}[1][]{\mathbf{\virval}_{#1}}
\newcommand{\virvald}[1][]{\psi_{#1}}
\newcommand{\virvaldvec}[1][]{\mathbf{\virvald}_{#1}}

\usepackage[framemethod=tikz]{mdframed}
\definecolor{mycolor}{rgb}{0.122, 0.435, 0.698}
\newmdenv[innerlinewidth=0.5pt, roundcorner=4pt,linecolor=mycolor,innerleftmargin=6pt,
innerrightmargin=6pt,innertopmargin=6pt,innerbottommargin=6pt]{mybox}

\newcommand{\commente}[1]{\textbf{[Enrique: #1]}}
\newcommand{\commenta}[1]{\textbf{[Amy: #1]}}
\newcommand{\commentb}[1]{\textbf{[Betsy: #1]}}

%%%%%%%%%%

\title{{\it AdX Game\/}: The Ad Exchange Game}

\author{Enrique Areyan Viqueira and Amy Greenwald}

\DeclarePairedDelimiter\ceil{\lceil}{\rceil}
\DeclarePairedDelimiter\floor{\lfloor}{\rfloor}

%%%%